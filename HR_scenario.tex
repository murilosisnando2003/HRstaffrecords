% Options for packages loaded elsewhere
\PassOptionsToPackage{unicode}{hyperref}
\PassOptionsToPackage{hyphens}{url}
%
\documentclass[
]{article}
\usepackage{lmodern}
\usepackage{amssymb,amsmath}
\usepackage{ifxetex,ifluatex}
\ifnum 0\ifxetex 1\fi\ifluatex 1\fi=0 % if pdftex
  \usepackage[T1]{fontenc}
  \usepackage[utf8]{inputenc}
  \usepackage{textcomp} % provide euro and other symbols
\else % if luatex or xetex
  \usepackage{unicode-math}
  \defaultfontfeatures{Scale=MatchLowercase}
  \defaultfontfeatures[\rmfamily]{Ligatures=TeX,Scale=1}
\fi
% Use upquote if available, for straight quotes in verbatim environments
\IfFileExists{upquote.sty}{\usepackage{upquote}}{}
\IfFileExists{microtype.sty}{% use microtype if available
  \usepackage[]{microtype}
  \UseMicrotypeSet[protrusion]{basicmath} % disable protrusion for tt fonts
}{}
\makeatletter
\@ifundefined{KOMAClassName}{% if non-KOMA class
  \IfFileExists{parskip.sty}{%
    \usepackage{parskip}
  }{% else
    \setlength{\parindent}{0pt}
    \setlength{\parskip}{6pt plus 2pt minus 1pt}}
}{% if KOMA class
  \KOMAoptions{parskip=half}}
\makeatother
\usepackage{xcolor}
\IfFileExists{xurl.sty}{\usepackage{xurl}}{} % add URL line breaks if available
\IfFileExists{bookmark.sty}{\usepackage{bookmark}}{\usepackage{hyperref}}
\hypersetup{
  pdftitle={HR data: staff records},
  pdfauthor={Murilo Henrique Sisnando Rodrigues},
  hidelinks,
  pdfcreator={LaTeX via pandoc}}
\urlstyle{same} % disable monospaced font for URLs
\usepackage[margin=1in]{geometry}
\usepackage{color}
\usepackage{fancyvrb}
\newcommand{\VerbBar}{|}
\newcommand{\VERB}{\Verb[commandchars=\\\{\}]}
\DefineVerbatimEnvironment{Highlighting}{Verbatim}{commandchars=\\\{\}}
% Add ',fontsize=\small' for more characters per line
\usepackage{framed}
\definecolor{shadecolor}{RGB}{248,248,248}
\newenvironment{Shaded}{\begin{snugshade}}{\end{snugshade}}
\newcommand{\AlertTok}[1]{\textcolor[rgb]{0.94,0.16,0.16}{#1}}
\newcommand{\AnnotationTok}[1]{\textcolor[rgb]{0.56,0.35,0.01}{\textbf{\textit{#1}}}}
\newcommand{\AttributeTok}[1]{\textcolor[rgb]{0.77,0.63,0.00}{#1}}
\newcommand{\BaseNTok}[1]{\textcolor[rgb]{0.00,0.00,0.81}{#1}}
\newcommand{\BuiltInTok}[1]{#1}
\newcommand{\CharTok}[1]{\textcolor[rgb]{0.31,0.60,0.02}{#1}}
\newcommand{\CommentTok}[1]{\textcolor[rgb]{0.56,0.35,0.01}{\textit{#1}}}
\newcommand{\CommentVarTok}[1]{\textcolor[rgb]{0.56,0.35,0.01}{\textbf{\textit{#1}}}}
\newcommand{\ConstantTok}[1]{\textcolor[rgb]{0.00,0.00,0.00}{#1}}
\newcommand{\ControlFlowTok}[1]{\textcolor[rgb]{0.13,0.29,0.53}{\textbf{#1}}}
\newcommand{\DataTypeTok}[1]{\textcolor[rgb]{0.13,0.29,0.53}{#1}}
\newcommand{\DecValTok}[1]{\textcolor[rgb]{0.00,0.00,0.81}{#1}}
\newcommand{\DocumentationTok}[1]{\textcolor[rgb]{0.56,0.35,0.01}{\textbf{\textit{#1}}}}
\newcommand{\ErrorTok}[1]{\textcolor[rgb]{0.64,0.00,0.00}{\textbf{#1}}}
\newcommand{\ExtensionTok}[1]{#1}
\newcommand{\FloatTok}[1]{\textcolor[rgb]{0.00,0.00,0.81}{#1}}
\newcommand{\FunctionTok}[1]{\textcolor[rgb]{0.00,0.00,0.00}{#1}}
\newcommand{\ImportTok}[1]{#1}
\newcommand{\InformationTok}[1]{\textcolor[rgb]{0.56,0.35,0.01}{\textbf{\textit{#1}}}}
\newcommand{\KeywordTok}[1]{\textcolor[rgb]{0.13,0.29,0.53}{\textbf{#1}}}
\newcommand{\NormalTok}[1]{#1}
\newcommand{\OperatorTok}[1]{\textcolor[rgb]{0.81,0.36,0.00}{\textbf{#1}}}
\newcommand{\OtherTok}[1]{\textcolor[rgb]{0.56,0.35,0.01}{#1}}
\newcommand{\PreprocessorTok}[1]{\textcolor[rgb]{0.56,0.35,0.01}{\textit{#1}}}
\newcommand{\RegionMarkerTok}[1]{#1}
\newcommand{\SpecialCharTok}[1]{\textcolor[rgb]{0.00,0.00,0.00}{#1}}
\newcommand{\SpecialStringTok}[1]{\textcolor[rgb]{0.31,0.60,0.02}{#1}}
\newcommand{\StringTok}[1]{\textcolor[rgb]{0.31,0.60,0.02}{#1}}
\newcommand{\VariableTok}[1]{\textcolor[rgb]{0.00,0.00,0.00}{#1}}
\newcommand{\VerbatimStringTok}[1]{\textcolor[rgb]{0.31,0.60,0.02}{#1}}
\newcommand{\WarningTok}[1]{\textcolor[rgb]{0.56,0.35,0.01}{\textbf{\textit{#1}}}}
\usepackage{graphicx}
\makeatletter
\def\maxwidth{\ifdim\Gin@nat@width>\linewidth\linewidth\else\Gin@nat@width\fi}
\def\maxheight{\ifdim\Gin@nat@height>\textheight\textheight\else\Gin@nat@height\fi}
\makeatother
% Scale images if necessary, so that they will not overflow the page
% margins by default, and it is still possible to overwrite the defaults
% using explicit options in \includegraphics[width, height, ...]{}
\setkeys{Gin}{width=\maxwidth,height=\maxheight,keepaspectratio}
% Set default figure placement to htbp
\makeatletter
\def\fps@figure{htbp}
\makeatother
\setlength{\emergencystretch}{3em} % prevent overfull lines
\providecommand{\tightlist}{%
  \setlength{\itemsep}{0pt}\setlength{\parskip}{0pt}}
\setcounter{secnumdepth}{-\maxdimen} % remove section numbering

\title{HR data: staff records}
\author{Murilo Henrique Sisnando Rodrigues}
\date{}

\begin{document}
\maketitle

\hypertarget{the-scenario}{%
\subsection{The scenario:}\label{the-scenario}}

The fictional wholesale company, CR25 (for ``Classic Rock 25''),
currently has 999 employees.

\hypertarget{employee-characteristics}{%
\subsubsection{employee
characteristics}\label{employee-characteristics}}

The CSV file ``df\_HR\_main.csv'' contains the following variables:

\begin{itemize}
\item
  emp\_id -- a unique identification number for each employee
\item
  start\_date -- the date the employee started their employment with
  CR25
\item
  four variables based on the designated groups specified in Canada's
  \href{https://laws-lois.justice.gc.ca/eng/acts/E-5.401/}{\emph{Employment
  Equity Act}}, where ``TRUE'' indicates the the employee is a member of
  that group

  \begin{itemize}
  \item
    women
  \item
    Aboriginal peoples
  \item
    persons with disabilities
  \item
    members of visible minorities (meaning ``persons, other than
    Aboriginal peoples, who are non-Caucasian in race or non-white in
    colour'').
  \end{itemize}
\end{itemize}

\hypertarget{transaction-history}{%
\subsubsection{transaction history}\label{transaction-history}}

Some of the employees in this company have been with the company since
it started on 2010-01-01, and others started recently.

There are 5 occupations in the company, labelled A (for Apprentice) to E
(for Executive), and each has more responsibility (and therefore pay)
than the one below. Employees can start at any one of the occupations,
and are eligible for a promotion after 90 days on the job (but some
people take longer than that).

Each occupation has 3 steps (numbered 1 to 3); everyone starts at step
1, and moves to step 2 after 1 year in that occupation, and step 3 after
2 years.

The second CSV file ``df\_HR\_transaction.csv'' contains the promotion
history for each employee. The variables are:

\begin{itemize}
\item
  emp\_id -- a unique identification number for each employee
\item
  transaction\_num -- the hiring and promotion event categories
\item
  transaction\_date -- the date of the hiring and promotion event
\item
  occupation -- the five occupation categories
\item
  occ\_step -- the annual increment steps
\end{itemize}

\hypertarget{your-taks}{%
\subsubsection{Your taks}\label{your-taks}}

You work in the human resources department, and need to make a summary
for the company's annual meeting. You will need to present tables and
charts showing the hiring and promotion of staff within the occupational
categories. To do this you will present summaries of the data with plots
and tables.

\hypertarget{project-starts-here}{%
\subsubsection{Project Starts here:}\label{project-starts-here}}

\hypertarget{loading-libraries}{%
\paragraph{1. Loading Libraries}\label{loading-libraries}}

\begin{Shaded}
\begin{Highlighting}[]
\KeywordTok{library}\NormalTok{(tidyverse)}
\end{Highlighting}
\end{Shaded}

\begin{verbatim}
## -- Attaching packages --------------------------------------- tidyverse 1.3.0 --
\end{verbatim}

\begin{verbatim}
## v ggplot2 3.3.2     v purrr   0.3.4
## v tibble  3.0.4     v dplyr   1.0.2
## v tidyr   1.1.2     v stringr 1.4.0
## v readr   1.4.0     v forcats 0.5.0
\end{verbatim}

\begin{verbatim}
## -- Conflicts ------------------------------------------ tidyverse_conflicts() --
## x dplyr::filter() masks stats::filter()
## x dplyr::lag()    masks stats::lag()
\end{verbatim}

\begin{Shaded}
\begin{Highlighting}[]
\KeywordTok{library}\NormalTok{(dplyr)}

\KeywordTok{library}\NormalTok{(scales) }
\end{Highlighting}
\end{Shaded}

\begin{verbatim}
## 
## Attaching package: 'scales'
\end{verbatim}

\begin{verbatim}
## The following object is masked from 'package:purrr':
## 
##     discard
\end{verbatim}

\begin{verbatim}
## The following object is masked from 'package:readr':
## 
##     col_factor
\end{verbatim}

\begin{Shaded}
\begin{Highlighting}[]
\KeywordTok{library}\NormalTok{(lubridate)}
\end{Highlighting}
\end{Shaded}

\begin{verbatim}
## 
## Attaching package: 'lubridate'
\end{verbatim}

\begin{verbatim}
## The following objects are masked from 'package:base':
## 
##     date, intersect, setdiff, union
\end{verbatim}

\begin{Shaded}
\begin{Highlighting}[]
\KeywordTok{library}\NormalTok{ (janitor)}
\end{Highlighting}
\end{Shaded}

\begin{verbatim}
## 
## Attaching package: 'janitor'
\end{verbatim}

\begin{verbatim}
## The following objects are masked from 'package:stats':
## 
##     chisq.test, fisher.test
\end{verbatim}

\begin{Shaded}
\begin{Highlighting}[]
\KeywordTok{library}\NormalTok{(gganimate)}


\KeywordTok{library}\NormalTok{(vtree)}
\end{Highlighting}
\end{Shaded}

\hypertarget{importing-data-csv}{%
\paragraph{2. Importing data: csv}\label{importing-data-csv}}

\begin{Shaded}
\begin{Highlighting}[]
\NormalTok{HR\_main \textless{}{-}}\StringTok{ }\KeywordTok{read\_csv}\NormalTok{(}\StringTok{"df\_HR\_main.csv"}\NormalTok{)}
\end{Highlighting}
\end{Shaded}

\begin{verbatim}
## 
## -- Column specification --------------------------------------------------------
## cols(
##   emp_id = col_character(),
##   start_date = col_date(format = ""),
##   equity_woman = col_logical(),
##   equity_visible_minority = col_logical(),
##   equity_aboriginal = col_logical(),
##   equity_disability = col_logical()
## )
\end{verbatim}

\begin{Shaded}
\begin{Highlighting}[]
\NormalTok{HR\_transaction \textless{}{-}}\StringTok{ }\KeywordTok{read\_csv}\NormalTok{(}\StringTok{"df\_HR\_transaction.csv"}\NormalTok{)}
\end{Highlighting}
\end{Shaded}

\begin{verbatim}
## 
## -- Column specification --------------------------------------------------------
## cols(
##   emp_id = col_character(),
##   transaction_num = col_character(),
##   transaction_date = col_date(format = ""),
##   occupation = col_character(),
##   occ_step = col_double()
## )
\end{verbatim}

\begin{Shaded}
\begin{Highlighting}[]
\NormalTok{HR\_main}
\end{Highlighting}
\end{Shaded}

\begin{verbatim}
## # A tibble: 999 x 6
##    emp_id start_date equity_woman equity_visible_~ equity_aborigin~
##    <chr>  <date>     <lgl>        <lgl>            <lgl>           
##  1 ID001  2010-01-02 TRUE         FALSE            FALSE           
##  2 ID002  2010-01-03 TRUE         TRUE             TRUE            
##  3 ID003  2010-01-08 FALSE        FALSE            TRUE            
##  4 ID004  2010-01-10 TRUE         TRUE             FALSE           
##  5 ID005  2010-01-16 TRUE         FALSE            FALSE           
##  6 ID006  2010-01-17 TRUE         FALSE            FALSE           
##  7 ID007  2010-01-21 TRUE         FALSE            FALSE           
##  8 ID008  2010-01-25 FALSE        FALSE            FALSE           
##  9 ID009  2010-01-27 TRUE         FALSE            FALSE           
## 10 ID010  2010-01-30 TRUE         FALSE            FALSE           
## # ... with 989 more rows, and 1 more variable: equity_disability <lgl>
\end{verbatim}

\begin{Shaded}
\begin{Highlighting}[]
\NormalTok{HR\_transaction}
\end{Highlighting}
\end{Shaded}

\begin{verbatim}
## # A tibble: 2,997 x 5
##    emp_id transaction_num transaction_date occupation occ_step
##    <chr>  <chr>           <date>           <chr>         <dbl>
##  1 ID001  start_date      2010-01-02       D                 3
##  2 ID001  trans1          2019-12-20       E                 1
##  3 ID001  trans2          NA               <NA>             NA
##  4 ID002  start_date      2010-01-03       C                 3
##  5 ID002  trans1          2020-03-18       D                 1
##  6 ID002  trans2          2020-10-16       E                 1
##  7 ID003  start_date      2010-01-08       A                 3
##  8 ID003  trans1          2013-02-18       B                 1
##  9 ID003  trans2          2013-08-15       C                 3
## 10 ID004  start_date      2010-01-10       B                 3
## # ... with 2,987 more rows
\end{verbatim}

\textbf{\emph{Use \texttt{distinct} to check that
\texttt{HR\_main\$emp\_id} is a primary key}}

\begin{Shaded}
\begin{Highlighting}[]
\NormalTok{HR\_main }\OperatorTok{\%\textgreater{}\%}
\StringTok{  }\KeywordTok{distinct}\NormalTok{(HR\_main}\OperatorTok{$}\NormalTok{emp\_id)}
\end{Highlighting}
\end{Shaded}

\begin{verbatim}
## # A tibble: 999 x 1
##    `HR_main$emp_id`
##    <chr>           
##  1 ID001           
##  2 ID002           
##  3 ID003           
##  4 ID004           
##  5 ID005           
##  6 ID006           
##  7 ID007           
##  8 ID008           
##  9 ID009           
## 10 ID010           
## # ... with 989 more rows
\end{verbatim}

\textbf{\emph{Use \texttt{distinct} to check that
\texttt{HR\_transaction\$emp\_id} is a primary key}}

\begin{Shaded}
\begin{Highlighting}[]
\NormalTok{HR\_transaction }\OperatorTok{\%\textgreater{}\%}
\StringTok{  }\KeywordTok{distinct}\NormalTok{(HR\_transaction}\OperatorTok{$}\NormalTok{emp\_id)}
\end{Highlighting}
\end{Shaded}

\begin{verbatim}
## # A tibble: 999 x 1
##    `HR_transaction$emp_id`
##    <chr>                  
##  1 ID001                  
##  2 ID002                  
##  3 ID003                  
##  4 ID004                  
##  5 ID005                  
##  6 ID006                  
##  7 ID007                  
##  8 ID008                  
##  9 ID009                  
## 10 ID010                  
## # ... with 989 more rows
\end{verbatim}

\textbf{\emph{Using Lubridate to create a month/year column}}

\begin{Shaded}
\begin{Highlighting}[]
\NormalTok{HR\_main \textless{}{-}}\StringTok{ }\NormalTok{HR\_main }\OperatorTok{\%\textgreater{}\%}
\StringTok{  }\KeywordTok{mutate}\NormalTok{(}
    \DataTypeTok{Month\_Yr =} \KeywordTok{format\_ISO8601}\NormalTok{(start\_date, }\DataTypeTok{precision =} \StringTok{"ym"}\NormalTok{),}
    \DataTypeTok{Year =} \KeywordTok{format\_ISO8601}\NormalTok{(start\_date, }\DataTypeTok{precision =} \StringTok{"y"}\NormalTok{)}
    
\NormalTok{  )}

\NormalTok{HR\_main}
\end{Highlighting}
\end{Shaded}

\begin{verbatim}
## # A tibble: 999 x 8
##    emp_id start_date equity_woman equity_visible_~ equity_aborigin~
##    <chr>  <date>     <lgl>        <lgl>            <lgl>           
##  1 ID001  2010-01-02 TRUE         FALSE            FALSE           
##  2 ID002  2010-01-03 TRUE         TRUE             TRUE            
##  3 ID003  2010-01-08 FALSE        FALSE            TRUE            
##  4 ID004  2010-01-10 TRUE         TRUE             FALSE           
##  5 ID005  2010-01-16 TRUE         FALSE            FALSE           
##  6 ID006  2010-01-17 TRUE         FALSE            FALSE           
##  7 ID007  2010-01-21 TRUE         FALSE            FALSE           
##  8 ID008  2010-01-25 FALSE        FALSE            FALSE           
##  9 ID009  2010-01-27 TRUE         FALSE            FALSE           
## 10 ID010  2010-01-30 TRUE         FALSE            FALSE           
## # ... with 989 more rows, and 3 more variables: equity_disability <lgl>,
## #   Month_Yr <chr>, Year <chr>
\end{verbatim}

\textbf{\emph{Visualize - a draft plot}}

\begin{Shaded}
\begin{Highlighting}[]
\NormalTok{HR\_main }\OperatorTok{\%\textgreater{}\%}
\StringTok{   }\KeywordTok{group\_by}\NormalTok{(}\StringTok{\textasciigrave{}}\DataTypeTok{Year}\StringTok{\textasciigrave{}}\NormalTok{)}\OperatorTok{\%\textgreater{}\%}
\StringTok{   }\KeywordTok{summarise}\NormalTok{(}\DataTypeTok{count=}\KeywordTok{n}\NormalTok{())}\OperatorTok{\%\textgreater{}\%}
\StringTok{   }\KeywordTok{ggplot}\NormalTok{(}\KeywordTok{aes}\NormalTok{(}\DataTypeTok{x=}\NormalTok{Year, }\DataTypeTok{y=}\NormalTok{count, }\DataTypeTok{colour=}\NormalTok{count)) }\OperatorTok{+}\StringTok{ }
\StringTok{   }\KeywordTok{geom\_line}\NormalTok{(}\KeywordTok{aes}\NormalTok{(}\DataTypeTok{group=}\DecValTok{1}\NormalTok{)) }\OperatorTok{+}\StringTok{ }\KeywordTok{geom\_point}\NormalTok{() }\OperatorTok{+}\StringTok{ }\KeywordTok{theme\_bw}\NormalTok{() }\OperatorTok{+}
\StringTok{   }\KeywordTok{theme}\NormalTok{(}
    \DataTypeTok{panel.border =} \KeywordTok{element\_rect}\NormalTok{(}\DataTypeTok{colour=}\StringTok{"white"}\NormalTok{),}
    \DataTypeTok{plot.title =} \KeywordTok{element\_text}\NormalTok{(}\DataTypeTok{face=}\StringTok{"bold"}\NormalTok{),}
    \DataTypeTok{axis.line =} \KeywordTok{element\_line}\NormalTok{(}\DataTypeTok{colour=}\StringTok{"black"}\NormalTok{),}
    \DataTypeTok{axis.title =} \KeywordTok{element\_text}\NormalTok{(}\DataTypeTok{size=}\DecValTok{12}\NormalTok{),}
    \DataTypeTok{axis.text =} \KeywordTok{element\_text}\NormalTok{(}\DataTypeTok{size=}\DecValTok{12}\NormalTok{)}
\NormalTok{   )}\OperatorTok{+}
\StringTok{  }\KeywordTok{labs}\NormalTok{(}\DataTypeTok{title =} \StringTok{"Quantity of New Employees by Year"}\NormalTok{,}\DataTypeTok{x=} \StringTok{"Year"}\NormalTok{, }\DataTypeTok{y=} \StringTok{"Quantity of Employees"}\NormalTok{)}
\end{Highlighting}
\end{Shaded}

\begin{verbatim}
## `summarise()` ungrouping output (override with `.groups` argument)
\end{verbatim}

\includegraphics{HR_scenario_files/figure-latex/unnamed-chunk-6-1.pdf}

\textbf{\emph{Visualize - a draft plot}}

\begin{Shaded}
\begin{Highlighting}[]
\KeywordTok{theme\_set}\NormalTok{(}\KeywordTok{theme\_bw}\NormalTok{())}

\NormalTok{HR \textless{}{-}}\StringTok{ }\KeywordTok{inner\_join}\NormalTok{(HR\_main, HR\_transaction, }\DataTypeTok{key =} \StringTok{"emp\_id"}\NormalTok{)}\OperatorTok{\%\textgreater{}\%}
\StringTok{  }\KeywordTok{filter}\NormalTok{(}\OperatorTok{!}\KeywordTok{is.na}\NormalTok{(occupation))}\OperatorTok{\%\textgreater{}\%}
\StringTok{   }\KeywordTok{group\_by}\NormalTok{(}\StringTok{\textasciigrave{}}\DataTypeTok{occupation}\StringTok{\textasciigrave{}}\NormalTok{)}\OperatorTok{\%\textgreater{}\%}
\StringTok{   }\KeywordTok{summarise}\NormalTok{(}\DataTypeTok{count=}\KeywordTok{n}\NormalTok{())}\OperatorTok{\%\textgreater{}\%}
\StringTok{   }\CommentTok{\# Draw plot}
\StringTok{  }\KeywordTok{ggplot}\NormalTok{( }\KeywordTok{aes}\NormalTok{(}\DataTypeTok{x=}\NormalTok{occupation, }\DataTypeTok{y=}\NormalTok{count)) }\OperatorTok{+}\StringTok{ }
\StringTok{  }\KeywordTok{geom\_bar}\NormalTok{(}\DataTypeTok{stat=}\StringTok{"identity"}\NormalTok{, }\DataTypeTok{width=}\NormalTok{.}\DecValTok{5}\NormalTok{, }\DataTypeTok{fill=}\StringTok{"tomato3"}\NormalTok{) }\OperatorTok{+}\StringTok{ }
\StringTok{  }\KeywordTok{labs}\NormalTok{(}\DataTypeTok{title=}\StringTok{"Occupation Vs Employees {-} 2010 to 2020"}\NormalTok{, }
       \DataTypeTok{subtitle=}\StringTok{"Classic Rock 25 Company "}\NormalTok{, }
       \DataTypeTok{caption=}\StringTok{"source: HR Data from 2010 to 2020 "}\NormalTok{,}
       \DataTypeTok{x =} \StringTok{"Occupation"}\NormalTok{, }\DataTypeTok{y =} \StringTok{"Qty. Employees"}\NormalTok{ ) }\OperatorTok{+}\StringTok{ }
\StringTok{  }\KeywordTok{theme}\NormalTok{(}\DataTypeTok{axis.text.x =} \KeywordTok{element\_text}\NormalTok{(}\DataTypeTok{angle=}\DecValTok{65}\NormalTok{, }\DataTypeTok{vjust=}\FloatTok{0.6}\NormalTok{))}
\end{Highlighting}
\end{Shaded}

\begin{verbatim}
## Joining, by = "emp_id"
\end{verbatim}

\begin{verbatim}
## `summarise()` ungrouping output (override with `.groups` argument)
\end{verbatim}

\begin{Shaded}
\begin{Highlighting}[]
\NormalTok{HR}
\end{Highlighting}
\end{Shaded}

\includegraphics{HR_scenario_files/figure-latex/unnamed-chunk-7-1.pdf}

\begin{Shaded}
\begin{Highlighting}[]
\KeywordTok{theme\_set}\NormalTok{(}\KeywordTok{theme\_bw}\NormalTok{())}

\NormalTok{HR \textless{}{-}}\StringTok{ }\KeywordTok{inner\_join}\NormalTok{(HR\_main, HR\_transaction, }\DataTypeTok{key =} \StringTok{"emp\_id"}\NormalTok{)}\OperatorTok{\%\textgreater{}\%}
\KeywordTok{filter}\NormalTok{(}\OperatorTok{!}\KeywordTok{is.na}\NormalTok{(occupation),transaction\_num }\OperatorTok{==}\StringTok{ "start\_date"}\NormalTok{,Year}\OperatorTok{==}\DecValTok{2020}\NormalTok{)}\OperatorTok{\%\textgreater{}\%}
\StringTok{   }\KeywordTok{group\_by}\NormalTok{(}\StringTok{\textasciigrave{}}\DataTypeTok{occupation}\StringTok{\textasciigrave{}}\NormalTok{)}\OperatorTok{\%\textgreater{}\%}
\StringTok{   }\KeywordTok{summarise}\NormalTok{(}\DataTypeTok{count=}\KeywordTok{n}\NormalTok{())}\OperatorTok{\%\textgreater{}\%}
\StringTok{   }\CommentTok{\# Draw plot}
\StringTok{  }\KeywordTok{ggplot}\NormalTok{( }\KeywordTok{aes}\NormalTok{(}\DataTypeTok{x=}\NormalTok{occupation, }\DataTypeTok{y=}\NormalTok{count)) }\OperatorTok{+}\StringTok{ }
\StringTok{  }\KeywordTok{geom\_bar}\NormalTok{(}\DataTypeTok{stat=}\StringTok{"identity"}\NormalTok{, }\DataTypeTok{width=}\NormalTok{.}\DecValTok{5}\NormalTok{, }\DataTypeTok{fill=}\StringTok{"blue"}\NormalTok{) }\OperatorTok{+}\StringTok{ }
\StringTok{  }\KeywordTok{labs}\NormalTok{(}\DataTypeTok{title=}\StringTok{"New Employees by Occupation {-} 2020 "}\NormalTok{, }
       \DataTypeTok{subtitle=}\StringTok{"Occupation Vs Qty.Employees "}\NormalTok{, }
       \DataTypeTok{caption=}\StringTok{"source: HR Data "}\NormalTok{,}
       \DataTypeTok{x =} \StringTok{"Occupation"}\NormalTok{, }\DataTypeTok{y =} \StringTok{"Qty. Employees"}\NormalTok{ ) }\OperatorTok{+}\StringTok{ }
\StringTok{  }\KeywordTok{theme}\NormalTok{(}\DataTypeTok{axis.text.x =} \KeywordTok{element\_text}\NormalTok{(}\DataTypeTok{angle=}\DecValTok{65}\NormalTok{, }\DataTypeTok{vjust=}\FloatTok{0.6}\NormalTok{))}
\end{Highlighting}
\end{Shaded}

\begin{verbatim}
## Joining, by = "emp_id"
\end{verbatim}

\begin{verbatim}
## `summarise()` ungrouping output (override with `.groups` argument)
\end{verbatim}

\begin{Shaded}
\begin{Highlighting}[]
\NormalTok{HR}
\end{Highlighting}
\end{Shaded}

\includegraphics{HR_scenario_files/figure-latex/unnamed-chunk-8-1.pdf}

\begin{Shaded}
\begin{Highlighting}[]
\KeywordTok{theme\_set}\NormalTok{(}\KeywordTok{theme\_bw}\NormalTok{())}

\NormalTok{HR \textless{}{-}}\StringTok{ }\KeywordTok{inner\_join}\NormalTok{(HR\_main, HR\_transaction, }\DataTypeTok{key =} \StringTok{"emp\_id"}\NormalTok{)}\OperatorTok{\%\textgreater{}\%}
\KeywordTok{filter}\NormalTok{(}\OperatorTok{!}\KeywordTok{is.na}\NormalTok{(occupation),transaction\_num }\OperatorTok{==}\StringTok{ "start\_date"}\NormalTok{,Year}\OperatorTok{==}\DecValTok{2020}\NormalTok{)}\OperatorTok{\%\textgreater{}\%}
\StringTok{  }\KeywordTok{filter}\NormalTok{(equity\_woman }\OperatorTok{==}\StringTok{ "TRUE"} \OperatorTok{|}\StringTok{ }\NormalTok{equity\_visible\_minority}\OperatorTok{==}\StringTok{ "TRUE"} \OperatorTok{|}\StringTok{ }\NormalTok{equity\_aboriginal}\OperatorTok{==}\StringTok{ "TRUE"} \OperatorTok{|}\StringTok{ }\NormalTok{equity\_disability}\OperatorTok{==}\StringTok{ "TRUE"}\NormalTok{)}\OperatorTok{\%\textgreater{}\%}
\StringTok{   }\KeywordTok{group\_by}\NormalTok{(}\StringTok{\textasciigrave{}}\DataTypeTok{occupation}\StringTok{\textasciigrave{}}\NormalTok{)}\OperatorTok{\%\textgreater{}\%}
\StringTok{   }\KeywordTok{summarise}\NormalTok{(}\DataTypeTok{count=}\KeywordTok{n}\NormalTok{())}\OperatorTok{\%\textgreater{}\%}
\StringTok{   }\CommentTok{\# Draw plot}
\StringTok{  }\KeywordTok{ggplot}\NormalTok{( }\KeywordTok{aes}\NormalTok{(}\DataTypeTok{x=}\NormalTok{occupation, }\DataTypeTok{y=}\NormalTok{count)) }\OperatorTok{+}\StringTok{ }
\StringTok{  }\KeywordTok{geom\_point}\NormalTok{(}\KeywordTok{aes}\NormalTok{(}\DataTypeTok{col=}\NormalTok{occupation, }\DataTypeTok{size=}\NormalTok{count)) }\OperatorTok{+}
\StringTok{  }\KeywordTok{labs}\NormalTok{(}\DataTypeTok{title=}\StringTok{"New Employees by Occupation {-} Equity "}\NormalTok{, }
       \DataTypeTok{subtitle=}\StringTok{"Occupation Vs Qty.Employees "}\NormalTok{, }
       \DataTypeTok{caption=}\StringTok{"source: HR Data "}\NormalTok{,}
       \DataTypeTok{x =} \StringTok{"Occupation"}\NormalTok{, }\DataTypeTok{y =} \StringTok{"Qty. Employees"}\NormalTok{ ) }\OperatorTok{+}\StringTok{ }
\StringTok{  }\KeywordTok{theme}\NormalTok{(}\DataTypeTok{axis.text.x =} \KeywordTok{element\_text}\NormalTok{(}\DataTypeTok{angle=}\DecValTok{65}\NormalTok{, }\DataTypeTok{vjust=}\FloatTok{0.6}\NormalTok{))}
\end{Highlighting}
\end{Shaded}

\begin{verbatim}
## Joining, by = "emp_id"
\end{verbatim}

\begin{verbatim}
## `summarise()` ungrouping output (override with `.groups` argument)
\end{verbatim}

\begin{Shaded}
\begin{Highlighting}[]
\NormalTok{HR}
\end{Highlighting}
\end{Shaded}

\includegraphics{HR_scenario_files/figure-latex/unnamed-chunk-9-1.pdf}

\textbf{\emph{Percentage of New Equity employees by year}}

\begin{Shaded}
\begin{Highlighting}[]
\NormalTok{HR2 \textless{}{-}}\StringTok{ }\KeywordTok{inner\_join}\NormalTok{(HR\_main, HR\_transaction, }\DataTypeTok{key =} \StringTok{"emp\_id"}\NormalTok{)}\OperatorTok{\%\textgreater{}\%}
\KeywordTok{filter}\NormalTok{(}\OperatorTok{!}\KeywordTok{is.na}\NormalTok{(occupation),transaction\_num }\OperatorTok{==}\StringTok{ "start\_date"}\NormalTok{)}\OperatorTok{\%\textgreater{}\%}
\StringTok{  }\KeywordTok{filter}\NormalTok{(equity\_woman }\OperatorTok{==}\StringTok{ "TRUE"} \OperatorTok{|}\StringTok{ }\NormalTok{equity\_visible\_minority}\OperatorTok{==}\StringTok{ "TRUE"} \OperatorTok{|}\StringTok{ }\NormalTok{equity\_aboriginal}\OperatorTok{==}\StringTok{ "TRUE"} \OperatorTok{|}\StringTok{ }\NormalTok{equity\_disability}\OperatorTok{==}\StringTok{ "TRUE"}\NormalTok{)}\OperatorTok{\%\textgreater{}\%}
\StringTok{  }\KeywordTok{tabyl}\NormalTok{(Year, occupation)}\OperatorTok{\%\textgreater{}\%}
\StringTok{  }\KeywordTok{adorn\_percentages}\NormalTok{(}\StringTok{"col"}\NormalTok{) }\OperatorTok{\%\textgreater{}\%}
\StringTok{  }\KeywordTok{adorn\_pct\_formatting}\NormalTok{()}
\end{Highlighting}
\end{Shaded}

\begin{verbatim}
## Joining, by = "emp_id"
\end{verbatim}

\begin{Shaded}
\begin{Highlighting}[]
\NormalTok{HR2}
\end{Highlighting}
\end{Shaded}

\begin{verbatim}
##  Year     A     B     C     D     E
##  2010 11.1%  7.6% 11.1%  2.2%  5.9%
##  2011 12.0% 12.7%  8.5%  8.7%  5.9%
##  2012  6.2% 11.7%  9.8%  0.0%  0.0%
##  2013 12.5%  9.1% 10.6%  6.5%  5.9%
##  2014  8.7%  9.6%  9.4%  4.3%  0.0%
##  2015 12.5%  7.6%  5.1%  6.5%  5.9%
##  2016  7.2% 10.2% 10.6% 10.9%  0.0%
##  2017 11.5% 10.7%  8.9% 19.6%  0.0%
##  2018  6.7% 10.2% 12.8% 15.2%  5.9%
##  2019  7.2%  5.1%  8.5% 15.2% 29.4%
##  2020  4.3%  5.6%  4.7% 10.9% 41.2%
\end{verbatim}

\textbf{\emph{Quantity o promotions by Occupation}}

\begin{Shaded}
\begin{Highlighting}[]
\NormalTok{HR3 \textless{}{-}}\StringTok{ }\KeywordTok{inner\_join}\NormalTok{(HR\_main, HR\_transaction, }\DataTypeTok{key =} \StringTok{"emp\_id"}\NormalTok{)}\OperatorTok{\%\textgreater{}\%}
\KeywordTok{filter}\NormalTok{(}\OperatorTok{!}\KeywordTok{is.na}\NormalTok{(occupation) }\OperatorTok{\&}\StringTok{ }\OperatorTok{!}\KeywordTok{is.na}\NormalTok{(transaction\_date))}\OperatorTok{\%\textgreater{}\%}
\KeywordTok{filter}\NormalTok{(transaction\_num }\OperatorTok{==}\StringTok{ "trans1"} \OperatorTok{|}\StringTok{ }\NormalTok{transaction\_num }\OperatorTok{==}\StringTok{ "trans1"}\NormalTok{)}\OperatorTok{\%\textgreater{}\%}
\StringTok{  }\KeywordTok{group\_by}\NormalTok{(}\StringTok{\textasciigrave{}}\DataTypeTok{occupation}\StringTok{\textasciigrave{}}\NormalTok{)}\OperatorTok{\%\textgreater{}\%}
\StringTok{   }\KeywordTok{summarise}\NormalTok{(}\DataTypeTok{count=}\KeywordTok{n}\NormalTok{())}\OperatorTok{\%\textgreater{}\%}
\StringTok{   }\CommentTok{\# Draw plot}
\StringTok{  }\KeywordTok{ggplot}\NormalTok{(}\KeywordTok{aes}\NormalTok{(}\DataTypeTok{x=}\NormalTok{occupation, }\DataTypeTok{y=}\NormalTok{count, }\DataTypeTok{colour=}\NormalTok{occupation)) }\OperatorTok{+}\StringTok{ }
\StringTok{   }\KeywordTok{geom\_line}\NormalTok{(}\KeywordTok{aes}\NormalTok{(}\DataTypeTok{group=}\DecValTok{1}\NormalTok{)) }\OperatorTok{+}\StringTok{ }\KeywordTok{geom\_point}\NormalTok{() }\OperatorTok{+}\StringTok{ }\KeywordTok{theme\_bw}\NormalTok{() }\OperatorTok{+}
\StringTok{   }\KeywordTok{theme}\NormalTok{(}
    \DataTypeTok{panel.border =} \KeywordTok{element\_rect}\NormalTok{(}\DataTypeTok{colour=}\StringTok{"white"}\NormalTok{),}
    \DataTypeTok{plot.title =} \KeywordTok{element\_text}\NormalTok{(}\DataTypeTok{face=}\StringTok{"bold"}\NormalTok{),}
    \DataTypeTok{axis.line =} \KeywordTok{element\_line}\NormalTok{(}\DataTypeTok{colour=}\StringTok{"black"}\NormalTok{),}
    \DataTypeTok{axis.title =} \KeywordTok{element\_text}\NormalTok{(}\DataTypeTok{size=}\DecValTok{12}\NormalTok{),}
    \DataTypeTok{axis.text =} \KeywordTok{element\_text}\NormalTok{(}\DataTypeTok{size=}\DecValTok{12}\NormalTok{)}
\NormalTok{   )}\OperatorTok{+}
\StringTok{  }\KeywordTok{labs}\NormalTok{(}\DataTypeTok{title =} \StringTok{"Quantity of Promotions x Occupation"}\NormalTok{,}\DataTypeTok{x=} \StringTok{"Occupation"}\NormalTok{, }\DataTypeTok{y=} \StringTok{"Quantity of Promotions"}\NormalTok{)}
\end{Highlighting}
\end{Shaded}

\begin{verbatim}
## Joining, by = "emp_id"
\end{verbatim}

\begin{verbatim}
## `summarise()` ungrouping output (override with `.groups` argument)
\end{verbatim}

\begin{Shaded}
\begin{Highlighting}[]
\NormalTok{HR3}
\end{Highlighting}
\end{Shaded}

\includegraphics{HR_scenario_files/figure-latex/unnamed-chunk-11-1.pdf}

\textbf{\emph{Quantity of promotions per year}}

\begin{Shaded}
\begin{Highlighting}[]
\NormalTok{HR3 \textless{}{-}}\StringTok{ }\KeywordTok{inner\_join}\NormalTok{(HR\_main, HR\_transaction, }\DataTypeTok{key =} \StringTok{"emp\_id"}\NormalTok{)}\OperatorTok{\%\textgreater{}\%}
\KeywordTok{filter}\NormalTok{(}\OperatorTok{!}\KeywordTok{is.na}\NormalTok{(occupation) }\OperatorTok{\&}\StringTok{ }\OperatorTok{!}\KeywordTok{is.na}\NormalTok{(transaction\_date))}\OperatorTok{\%\textgreater{}\%}
\KeywordTok{filter}\NormalTok{(transaction\_num }\OperatorTok{==}\StringTok{ "trans1"} \OperatorTok{|}\StringTok{ }\NormalTok{transaction\_num }\OperatorTok{==}\StringTok{ "trans1"}\NormalTok{)}\OperatorTok{\%\textgreater{}\%}
\StringTok{  }\KeywordTok{tabyl}\NormalTok{(Year, occupation)}
\end{Highlighting}
\end{Shaded}

\begin{verbatim}
## Joining, by = "emp_id"
\end{verbatim}

\begin{Shaded}
\begin{Highlighting}[]
\NormalTok{HR3}
\end{Highlighting}
\end{Shaded}

\begin{verbatim}
##  Year  B  C  D E
##  2010 33 25 31 1
##  2011 31 27 32 4
##  2012 29 35 30 1
##  2013 38 29 31 3
##  2014 27 28 28 2
##  2015 31 26 20 3
##  2016 22 26 30 5
##  2017 35 22 27 9
##  2018 22 30 31 8
##  2019 16 13 23 9
##  2020  8  6  9 2
\end{verbatim}

\end{document}
